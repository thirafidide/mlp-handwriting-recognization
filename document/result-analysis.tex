\section{Eksperimen}

	\subsection{Persiapan Eksperimen}
	Eksperimen akkan menggunakan Multilayer Perceptron yang telah diimplementasi, serta menggunakan dataset yang diberikan dari scele. Dataset terdiri dari 10.000 data
	dan 785 atribut, dengan atribut pertama adalah label, berisi angka 0-9, dan 784 atribut merepresentasikan \textit{pixel} dari gambar tulisan tangan label tersebut.
	Setiap \textit{pixel} berisi angka 0-255, merupakan gradasi putih (0) hingga hitam (255). Sebelum digunakan, dimensi data direduksi dengan menghapus atribut yang 
	memiliki nilai yang sama di setiap data. Hasil pengolahan menghapus 109 atribut dan menyisakan 676 atribut. 

	Implementasi MLP yang digunakan akan menerima 675 input piksel, dan mengeluarkan 10 output yang merepresentasikan label. Input dan Output MLP adalah bilangan riil 
	antara 0 hingga 1, sehingga dataset dinormalsasi menjadi bilangan antara 0 hingga 1. Atribut label dibagi menjadi 10, menjadi label\_0 hingga label\_9, yang berisi
	0 atau 1. Atribut piksel dinormalisasi dengan cara membaginya dengan 255.

	\subsection{Skema Perbandingan}
	Eksperimen akan dilakukan dengan membandingkan banyaknya unit di \textit{hidden layer} dan dampaknya terhadap \textit{error} hasil learning Multilayer Perceptron. 
	Perbandingan akan dilakukan dengan menggunakan model \textit{Exhaustive leave-one-out cross-validation} (LOOCV). Sebanyak 10.000 data akan dibagi menjadi 10 partisi
	dengan masing-masing 1.000 data, dan untuk setiap partisi akan menjadi data testing dan sisanya menjadi data training. Error hasil learning untuk setiap partisi akan dirata-rata.

	\subsection{Hasil dan Analisis}
	\lipsum[1]